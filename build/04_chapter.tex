\chapter{Problems, alternatives and possible
improvements}\label{problems-alternatives-and-possible-improvements}

Some alternative approaches were also considered during the designing.
One of them was not to use periodical checking at all. It was based on
the idea, that there would exist permanent connection with each neighbor
and the change of state would be indicated by the events related to this
connection. However it was rejected due to the requirements connected
with keeping the connection. Furthermore, the changes of ready state of
the node would have to be check either periodically or the node would
have to inform all its neighbors (even potential) about each change
which would lead to another problems to deal with and is in contrast
with passivity of the slaves anyway.

Another issue was whether use some more sophisticated way of
distributing the chunks. Namely some kind of hierarchy was considered
when the chunk references would be distributed to neighbors in packs,
where it would be further split and distributed among neighbor's
neighbors and so on, so a kind of a tree structure would be formed. The
transfer of the file would be processed directly between the initiator
and the leaf node. However this approach turned out to be complicated
and brings many problems. For example in case of failure of some node
which is high in the hierarchy a lot of chunks would have to be
re-distributed. Also the initiator's ability to control the distribution
would be reduced. Moreover, the advantages of this approach are not so
significant at all, because the biggest portion of time is spent during
transfers and processing the chunks and the time spent with distribution
of references is not important at all.

Interesting alternative would be use of the
anycast\footnote{https://en.wikipedia.org/wiki/Anycast} mechanism
provided by the IPv6 protocol. The nodes would be addressed with an
anycast address and each chunk would be simply sent to this address.
Obvious disadvantage of this approach is the loss of the control of the
distribution. More sophisticated variant would use special node which
would maintain the list of free nodes and schedule the distribution.
However this would break the peer to peer paradigm because of the
centralization of control to one specific node.

Maybe the biggest unnecessary delay could appear when the whole process
waits for some lost chunk. This is partially solved by resending chunks
after the timeout. Also some form of redundancy could help. If some node
is processing more tasks sequentially while using approximately the same
set of neighbors, the framework could also determine optimal chunk size
to achieve good ratio of transfer and computing times (the chunk should
not be too small because of the delays tied with its distribution) while
minimizing the possible delay caused by waiting for re-encoding of some
lost chunk.

Also the current implementation creates a separate connection for each
data transfer. Alternatively, each chunk could be delivered and returned
using the same connection which would lessen the demands of the
communication. The connection's termination would also indicate problem
with the chunk's processing or the neighbor itself. But the connection
termination does not necessarily mean the failure of the process.
Because it is desirable to avoid needless re-encoding of the chunks,
this situation would has to be treated specially which would introduce
additional problems. Also this approach does not fit very well to the
current design in which each logical action is executed as the sequence
of commands and for each sequence there is a special connection.
