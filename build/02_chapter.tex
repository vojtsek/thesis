\chapter{Implementation}\label{implementation}

This chapter describes some of the used mechanisms in more detail. It
also introduces some classes and methods, but it is not supposed to
serve as a detailed and full documentation. The
Doxygen\footnote{\url{https://en.wikipedia.org/wiki/Doxygen}} software
has been used to generate documentation so the complete overview of the
code can be found in the attached
HTML\footnote{\url{https://en.wikipedia.org/wiki/HTML}} documents. The
program is implemented in the C++ language. This allows us to use the
standard C library functions. Also the functionality provided by the C++
STL\footnote{\url{https://en.wikipedia.org/wiki/Standard\_Template\_Library}}
is exploited. Some of its containers are used as a base for containers
with synchronized access implemented in the framework. Some of the
functionality from the
C++11\footnote{\url{https://en.wikipedia.org/wiki/C++}} standard is also
used, so the use of the program is limited to computers with compiler
supporting this standard.

\section{Networking}\label{networking}

One of the most important issue is how to handle the networking. The
chosen approach will be described in this section. As it has been said
already, the program uses TCP for network communication. This is
certainly good option when we want to handle data transfers, however, it
can be considered unnecessarily demanding for simple tasks. To preserve
the implementation simple, we chose not to use UDP. To utilize the
possibilities of the operating system, standard socket API is used.
Information about addresses are stored in the \textit{sockaddr\_storage}
structures which are suitable for storing both IPv4 and IPv6 addresses.
There are also some helper functions to work with address structures
which are generic in the use of an address family. To provide easy
manipulation with addresses, the structure \textit{MyAddr} was created
which groups the related functionality together. This approach makes it
possible to switch between both IP versions easily.

\subsection{Spawning connections}\label{spawning-connections}

The ultimate class to handle the connections is the
\textit{NetworkHandler} class. It provides all the necessary
functionality. Each instance of the program binds to the given listening
port and starts accepting connections. When the connection is accepted,
new thread is spawned to handle the connection. When the program wants
to make the connection, it provides structure referring to a given
neighbor together with the commands to be executed to the
\textit{NetworkHandler} instance. The connection is then spawned and
handled.

Here we encounter the topic of commands. Each action is represented by a
set of commands that implements it. Commands are instances of a class
which inherits the class \textit{Commnand}. Each command consists of two
parts. When one node initiates the connection, it provides the vector of
commands to be executed. They are processed in the loop in the following
manner: The command's \textit{execute} method is invoked. It
communicates over the network. First it sends name of the command to be
invoked in the peer node. The peer's thread loops too. First it reads
the name of the command and then it invokes it. In that moment there are
command methods running in both nodes and they can communicate. When
they end, the receiving node waits for another action. Meanwhile the
initiating node spawns next command from the vector. If the vector is
empty, the initiator ends the connection. This mechanism is used to
handle all the network communication. What happens in case of problems
is described in the section about handling errors.

Incoming connections are always handled asynchronously. Nevertheless,
the outgoing connections may be handled in the synchronous way. It can
be essential sometimes. For example if the node needs to obtain some
potential neighbors because his list is empty, it has to wait until the
action ends, because if it just sent the request and continued, it would
probably find the list still empty.

\subsection{Protocol}\label{protocol}

As it was said, the communication uses commands. Thread that handles the
incoming connection is provided with respective file descriptor and
address of the communicating node. First data that appears are
considered as the listening port number of the node, so it can be
identified in the neighbors list or added to the list of potential
neighbors. Then the name of the command is sent, which is represented by
the \textit{enum} type. If no error occurs, the confirmation is send and
the appropriate command's method is invoked. After returning from the
method, another command is read. If there are no data left, the
connection is closed. The most important commands are listed below.

\begin{enumerate}
\item {\large Commands for maintaining}
\begin{itemize}
\item  \textbf{Confirm} - Confirms the potential neighbor, adds the node to the neighbors list.
\item  \textbf{Ask} - Asks the neighbor for the list of addresses, receives the list and adds the addresses to the potential neighbors list.
\item  \textbf{Ping} - Verifies that the neighbor is alive, refreshes its status.
\item  \textbf{Cancel} - Cancels the request for particular chunk's encoding.
\item  \textbf{Goodbye} - Notifies the neighbor that the node itself is withdrawing from the network.
\end{itemize}
\item {\large Commands regarding transfers}
\begin{itemize}
\item  \textbf{Distribute} - Sends both the reference and the chunk itself during distribution.
\item  \textbf{Return} - Returns the encoded chunk back, together with the referencing structure, which is updated with the data about encoding and transfers.
\item  \textbf{Gather} - Spreads the request of the initiator to obtain more neighbors.
\end{itemize}
\end{enumerate}

\subsection{Transferring the data}\label{transferring-the-data}

First problem every network application has to deal with considers the
byte order. The POSIX sockets API provides set of functions to deal with
it. Namely it's \textit{htons} and \textit{ntohs}, or \textit{htonl} and
\textit{ntohl} respectively. These functions convert the native
representation of short (long) data types to the network byte order. The
framework uses these function.

\subsection*{Basic transfer functions}

Each of the functions mentioned in this paragraph has basically two
parts. One for sending and its receiving counterpart. Integers are
stored as the \textit{int64\_t} type which ensures correct communication
even between 32 and 64 bit nodes. They are sent using the mentioned
converting functions. When the string is transfered, first is sent the
length of the string, followed by the appropriate number of characters.
Commands are sent as numbers, wrapper functions are used which converts
between the \textit{enum} type and \textit{int32\_t} explicitly. Sending
the structures containing the addresses is managed by another special
function. It converts the address to string and sends it in this form,
followed by the port number. This allows to handle both IPv4 and IPv6
addresses, the format is recognized during the reversed conversion.

\subsection*{Transfering files}

The most delicate network task is to transfer the files. The file is
first check, and its size is determined. Then it is sent and then the
function repeatedly reads part of the file to buffer and sends it until
whole file is processed. The count of sent bytes is compared with the
actual file size in the end. The receiving side accepts the bytes and
writes it to the file continuously, the file size is checked in the end.
The data are first written to a temporary file which is renamed after
the successful transfer. This mechanism prevents inconsistency of the
received files. Each file is referenced by the
\hyperref[the-transferinfo-structure]{structure}. Among other things
this structure stores counter of unsuccessful sent tries. If the counter
exceeds given limit, the neighbor is treated as invalid and his state is
set to non-free. The file is also checked, if it is not valid for some
reason (the splitting process has encountered some error), the whole
process has to be aborted because there is no way how to fix one
specific chunk file.

\subsection{Handling errors}\label{handling-errors}

Unfortunately the network environment is quite error prone and all the
action has uncertain results. Moreover, the communication can be
interrupted at any time. Because of this it is important for the network
application to be able to deal with different error situations.

\subsection*{Errors during the connection handling}

Almost all the functions indicates error state by the negative return
code. These codes are checked so the error can propagate. If some error
happens during the control communication, the loop that handles the
connection simply brakes, so the connection is closed. The commands are
invoked in the try-catch block, so if the data have been corrupted or
the synchronization has been lost, the next invalid command name raises
an exception and the error is handled. The loss of synchronization may
be detected thanks to the obligatory confirmation of every command.

\subsection*{Other errors}

If an error happens during the file transfer, the receiving side detects
inconsistency thanks to checking of the file size, so the bad file can
be removed. Generally, if any error is encountered during the
communication, the corresponding execute method indicates it by its
return value, so it can be propagated further. The signal handler also
has to be set to cover situations when the connection is destroyed
unexpectedly and the
SIGPIPE\footnote{\url{https://en.wikipedia.org/wiki/Unix\_signal}} is
delivered.

\section{Structures' overview}\label{structures-overview}

This section describes two main structures that are used in the
framework. They common sign is, that they inherit from the
\textit{Listener} class, so they have to implement the invoke method.
This fact makes it possible to use them as
\hyperref[periodic-actions]{periodic listeners}.

\subsection{The TransferInfo
structure}\label{the-transferinfo-structure}

This structure serves for referencing the chunk. It contains flags used
for transfer, addresses (source and destination), information about the
video and path locating the physical position of the file. This field is
important because it makes it possible to reference the file. It is
changed several time during the process; as the state of the chunk
changes, it it located in various directories and it is important to
keep the value of the field actual. The structure also contains
information which help to determine the encoding process and some
statistics which describes the result. These are used to compute and
update the quality of the neighbor. The structure is also equipped with
a pair of methods that make it able to transfer it over the network.
This simplifies the usage of the structure. When the referenced chunk is
waiting for return, the corresponding method is invoked periodically. It
decreases the timer. For the first time the timer reaches zero the
neighbor which has this chunk assigned is checked. If it is alive, the
timer is set one more time. If the neighbor doesn't respond or the timer
reaches zero for the second time, the chunk is resent.

\subsection{The NeighborInfo
structure}\label{the-neighborinfo-structure}

Instances of this structure are kept in the
\hyperref[neighborstorage]{NeighborStorage class}. They represent the
neighbor, that is its address and listening port. It also keeps the
information about quality of the neighbor. The time elapsed from the
last check is stored too. Periodic invocation causes the timer to
decrease and possibly contact the neighbor to refresh the state.

\section{Important classes}\label{important-classes}

\subsection{The NeighborStorage class}\label{the-neighborstorage-class}

It is used for storing the \textit{NeighborInfo} structures. It provides
several methods to maintain neighbors list while preserving
synchronization. This is crucial, because the information about neighbor
can change any time but it's desirable to keep our knowledge consistent.
The class exists in one instance and helps to keep the information about
neighbors in one place, so the manipulation can be controlled.

\subsection{The NetworkHandler class}\label{the-networkhandler-class}

This class is used to handle all the networking issues. It provides
functionality to spawn the connections or contact neighbors. It also
holds the list of potential neighbors. This list differs from the
neighbors list, because besides address and port, which are necessary,
no further info is stored about potential neighbors. Also, most of the
other classes have not got a notion about this list. When the lack of
neighbors appears, simply the function \textit{obtainNeighbors} is
invoked, which uses the list internally. This class also handles adding
of the new neighbors.

\subsection{The Data class}\label{the-data-class}

It is a singleton class which helps to keep all the data at one place.
It also makes the data accessible from anywhere in the program. It holds
the instances of the \textit{NeighborStorage}, \textit{State} and all
the \hyperref[queues]{queues} that are used during the transfer and the
encoding process.

There are more significant classes such as TaskHandler and
WindowPrinter, which will be discussed in the corresponding sections.

\section{Periodic actions}\label{periodic-actions}

The framework uses a mechanism which invokes some actions periodically.
Pros and cons of this approach are described in the respective sections,
namely
\hyperref[problems-alternatives-and-possible-improvements]{the alternatives}.
To implement this mechanism, separate thread runs that loops and once
after each time quantum it invokes methods of structures that inherit
from the \textit{Listener} abstract class and are stored in the special
queue. The time quantum is defined as a constant, so all timers actually
express count of the quanta left. Obvious disadvantage of this approach
is busy waiting that is used in the loop. Alternative approach could use
a signal handler and setting an alarm. However, this could lead to not
necessary asynchronous interrupts. Since the mutexes are used to avoid
race conditions, a problem could occur if the signal interrupted some
method holding a ``bad'' mutex.

\section{Queues}\label{queues}

As it is described in the corresponding
\hyperref[distribution-of-chunks]{section}, the references to chunks can
appear in different queues. The chunk's placement depends on the state
in which it is. Since the whole process is nondeterministic, different
conditions may occur and more than one thread could need to work with
the queue at one moment. This means, that some way of serialization has
to be provided. Because of this, the
\textit{SynchronizedQueue structure} has been created. Basically it
provides usual functionality that could be requested from the queue, but
ensures handling race conditions because only one thread at a time can
access the underlying data. This is ensured by the \textit{mutex}. The
\textit{pop} method also uses the conditional variable, so if there are
no data that can be popped at the moment of invocation, it blocks and
waits for a signal.

\section{Working with input and
output}\label{working-with-input-and-output}

All the file operations are customized for use on the UNIX operating
system. Nevertheless, there is possibility to implement respective
functions to work on different operating systems easily.

\subsection*{File operations}

Each chunk is stored in a separate file. For this reason several
functions were implemented to allow easier work with the file system.
These include functions for manipulation with the filenames, controlling
files and working with directories. Also there is a generic function
which spawns an external process. This function uses process forking,
spawns the desired process and returns contents of its standard output
and standard error. The function also accepts value of timeout after
which it kills the process. This ensures, that it will not hang. The
result of the process' run propagates in the function's return value so
the caller can react accordingly. This mechanism is used to work with
video. Especially for splitting, encoding and joining it.

\subsection*{Working with the video}

The video processing is secured by the \textit{TaskHandler class}. When
it is loaded, some useful piece of information is obtained thanks to the
\textit{ffprobe} program. To allow easy processing, the output is in the
JSON\footnote{\url{https://en.wikipedia.org/wiki/JSON}} format which is
then parsed with the help of the
\textit{rapidjson}\footnote{\url{https://github.com/miloyip/rapidjson}}
library. This library consists of header files only and is distributed
as a part of the source code. It is available under the MIT license
which makes it suitable for our usage. Parsed values can be showed using
the F6 key. More importantly, they are used to compute the number of
chunks that will be created. Note, that the number of chunks can change
slightly after the splitting process. It is caused by the fact, that the
theoretically computed count does not consider the positions of key
frames. So the chunks can actually have different sizes and thus their
count differs. This fact also implies, that each chunk has got slightly
different size. When the task execution starts, the instance of
\textit{ffmpeg} program is spawned which splits the file. The result
files are then stored in the special subdirectory of the working
directory. For easy identification, each process has a unique code
assigned to it. This code is generated from the time stamp. The chunks
are numbered in increasing order. Chunk names are stored in the
referencing structures that are created after the split finishes. Then
the chunks are distributed as described in the
\hyperref[distribution-of-chunks]{section} about distributing. Each
processing neighbor encodes the chunks using the \textit{ffmpeg} and
then sends it back. The information about the chunk, such as the level
of the encoding quality is stored in the referencing structure. When all
the chunks are collected, a list of files to be joined is created. Then
the \textit{ffmpeg} is used again to join all these files to the output
file. The process can be aborted at any time. This action stops the
distribution, cleans the storages and notifies neighbors which are
processing some chunks so they can trash it.

An alternative approach to splitting the file was used during the
development. Firstly the position of every split was computed. Then it
was spawned one process per each chunk. The advantage was, that the
distribution could begin after the first chunk was created and therefore
some time was saved. However, this approach turned out to be bad because
of the existence of key frames. It does not allow to split the file at
the arbitrary position so certain shifts were observable in the result
file.

\subsection*{User interaction}

To provide interaction with the user, the \textit{curses} library is
used. User can provide input from the keyboard. There is set no delay of
the input, so the buffering is disabled. Because of this, we need to
handle the user input manually. On the other hand, it also makes it
possible to control the input and accept the commands immediately. The
output is provided using the \textit{WindowPrinter} class. This class
stores the queue of records and provides functionality to add or remove
some of them. Each record holds the line to be outputted together with
the style of the line. So each line can be displayed differently than
the others.

The screen is divided into four parts. Each part spans the whole width.
There is a line displaying available commands at the top. The biggest
portion of the vertical space belongs to two windows of equal height.
The first one displays different information about processing, neighbors
or file properties. The second window displays the status changes,
notifications and potentially some debugging messages. The bottom part
shows a prompt when user input is required.

Because there is usually a lot of threads that can cause the screen to
refresh (this means the particular \textit{WindowPrinter} instance is
updated), it is important to allow only one graphical update at the
moment, otherwise it could cause inconsistency of the graphical data.
This is ensured by a mutex assigned to each \textit{WindowPrinter}
instance.

\section{Synchronization}\label{synchronization}

Because of the nondeterministic nature of the application, it is
necessary to provide some kind of synchronization to ensure data and
information consistency. Mutex and conditional variable templates,
available in the C++ standard library, helps to deal with this issue. We
created implementations of queue and map like structures which ensure
serialized access. These classes use containers from the STL and the
synchronization primitives mentioned above. Namely they are
\textit{SynchronizedMap} and \textit{SynchronizedQueue} and are used to
store chunk references. Operations with the list of neighbors have to be
synchronized too. For example a race condition could occur, when one
thread would be working with the neighbor's reference while other would
want to remove the same neighbor from the list.

Another area which have to deal with some race conditions is output
which is displayed on the screen. The output is handled by the
\textit{curses} library which provides practically raw access to the
graphical data. This means, that if more threads try to work with the
screen at one moment, there is high possibility that they would
compromise each other and nonsensical data would be displayed at the
output. Moreover, this situation can also possibly result in the
segmentation fault. To avoid these situations, the data that are
supposed to appear on the screen are stored in the respective instances
of the \textit{WindowPrinter}. Mutexes are used to allow only one thread
to change the content of the storage or refresh the screen. This
approach has a disadvantage, that each call of the routine that produces
some output could be blocking.

\section{Error detection and
recovery}\label{error-detection-and-recovery}

During the process, various types of errors can occur. Errors connected
with the networking are discussed in the respective
\hyperref[handling-errors]{section}. Here we will describe other errors
that could possibly happen.

\subsection*{Neighbor failure}

If an unexpected failure of some node occurs, the other nodes which has
it in their lists must react. If the communication is interrupted in the
middle, there is no way how the other node can recognize the failure, so
the command's execution just fails. Nevertheless, this usually leads to
repetition of the command. This is able to register the failure.
Generally, the failure is noted when the try to establish a connection
with the given neighbor fails. It can occur during processing a command,
checking a neighbor or when the timer assigned with some chunk reaches
zero. In every of these situations the same function is used, so the
situation is always treated in the same way. The unresponsive neighbor
is removed from the list. If it has some chunks assigned, i.e.~they has
been sent to it already, they are queued for send to another neighbor.
If it has sent some chunks to be processed by the current node, those
chunks are trashed, because there will not be any use for them, since
there is no neighbor they should be returned to.

\subsection*{Chunk disappearance}

After the chunk is sent to be processed, it is pushed into the queue
which invokes its members periodically. When the time is up for the
first time, the respective neighbor is checked. If it responses, the
timeout is set again. If it does not respond, or the timer reaches zero
for the second time, the chunk is queued for sent. Also, in case of the
neighbor failure, the neighbor is removed. This can lead to a situation,
when one chunk is being processed by two different neighbors at the same
time. However, after it returns for the first time, it is put into the
dedicated storage, so if it returns afterwards, it is simply rejected.
But this situation should not occur, because when the chunk returns, all
neighbors that have it assigned are notified. They can cancel the
computation then. It can also occur the situation, when one node
receives by accident one chunk more times. The files are checked and
what is more, the transfer uses temporary files so the worst scenario
involves wasteful encoding of the chunk for the second time. For this
reason, each node remembers all the chunks it has processed, so this can
be avoided. It is important to store only the successfully processed
ones, because if the chunk was damaged during the transfer, the
initiator could ask for its repeated encoding and it would be valid in
this case. Another issue which has to be solved is how to set the
timeout. When the neighbor is involved in the computation for the first
time, we have no information about its performance. The timeout is thus
set respectively to the size of the chunk, default multiplication factor
is used. When the neighbor has already quality factor assigned, it is
used to compute the timeout. So the quality coefficient can be seen as
time needed to encode and transfer some unit of data. This coefficient
is recomputed with every chunk delivered by the respective neighbor.

\subsection*{Other errors}

Errors can also be encountered during the manipulation with the video.
Because all the video related problems are manipulated by the external
programs, the mechanism is used, which can control the process. The
approximate upper bounds are set for each task that is supposed to be
executed and if the process' execution takes too long, the signal is
sent that terminates the process. Then the error code is returned.
