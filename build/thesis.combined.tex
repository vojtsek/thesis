 
%%% The main file. It contains definitions of basic parameters and includes all other parts.

%% Settings for single-side (simplex) printing
% Margins: left 40mm, right 25mm, top and bottom 25mm
% (but beware, LaTeX adds 1in implicitly)
\documentclass[12pt,a4paper]{report}
\setlength\textwidth{145mm}
\setlength\textheight{247mm}
\setlength\oddsidemargin{15mm}
\setlength\evensidemargin{15mm}
\setlength\topmargin{0mm}
\setlength\headsep{0mm}
\setlength\headheight{0mm}
% Recommended layout mentions line spacing 1.5, but this is not relevant to TeX.
% \openright makes the following text appear on a right-hand page
\let\openright=\clearpage

%% Settings for two-sided (duplex) printing
% \documentclass[12pt,a4paper,twoside,openright]{report}
% \setlength\textwidth{145mm}
% \setlength\textheight{247mm}
% \setlength\oddsidemargin{14.2mm}
% \setlength\evensidemargin{0mm}
% \setlength\topmargin{0mm}
% \setlength\headsep{0mm}
% \setlength\headheight{0mm}
% \let\openright=\cleardoublepage

%% Character encoding: usually latin2, cp1250 or utf8:
\usepackage[utf8]{inputenc}

%% Further packages
\usepackage{graphicx}
\usepackage{amsthm}

%% Bibliography
\usepackage{natbib}
\usepackage{url}

%%% Basic information on the thesis

% Thesis title in English (exactly as in the formal assignment)
\def\ThesisTitle{Thesis title}

% Author of the thesis
\def\ThesisAuthor{Vojtech Hudecek}

% Year when the thesis is submitted
\def\YearSubmitted{2015}

% Name of the department or institute, where the work was officially assigned
% (according to the Organizational Structure of MFF UK in English,
% or a full name of a department outside MFF)
\def\Department{Department of Distributed and Depandable Systems}

% Is it a department (katedra), or an institute (ústav)?
\def\DeptType{Department}

% Thesis supervisor: name, surname and titles
\def\Supervisor{Supervisor's Name}

% Supervisor's department (again according to Organizational structure of MFF)
\def\SupervisorsDepartment{Department of Distributed and Depandable Systems}

% Study programme and specialization
\def\StudyProgramme{Computer Science}
\def\StudyBranch{General Computer Science}

% An optional dedication: you can thank whomever you wish (your supervisor,
% consultant, a person who lent the software, etc.)
\def\Dedication{%
Dedication.
}

% Abstract (recommended length around 80-200 words; this is not a copy of your thesis assignment!)
\def\Abstract{%
Despite today's computers' performance
there still exist some tasks that are quite time demanding. Nature of some
of these tasks allows to split them into smaller parts that can be processed in parallel.
Distributing work among more computers in order to speed up the process is a common
technique. However, most of the approaches use client-server architecture
to achieve this goal. We provide purely peer-to-peer solution which
allows high level of scalability, error recovery and easy maintaining.
No special role is needed in our framework and each node can join the network in any time. Also
the system is able to deal with node failures, keeping the overall computation time reasonable.
Tests showed that significant improvement can be achieved in local area networks.
}

% 3 to 5 keywords (recommended), each enclosed in curly braces
\def\Keywords{%
{Parallelization} {Peer to Peer} {Distributed computing} {Video encoding}
} 


%% The hyperref package for clickable links in PDF and also for storing
%% metadata to PDF (including the table of contents).
\usepackage[pdftex,unicode]{hyperref}   % Must follow all other packages
\hypersetup{breaklinks=true}
\hypersetup{pdftitle={\ThesisTitle}}
\hypersetup{pdfauthor={\ThesisAuthor}}
\hypersetup{pdfkeywords=\Keywords}

\input title.tex

%%% A page with automatically generated content of the bachelor thesis. For
%%% a mathematical thesis, it is permissible to have a list of tables and abbreviations,
%%% if any, at the beginning of the thesis instead of at its end.

\tableofcontents

%%% Each chapter is kept in a separate file
\chapter*{Introduction}
\addcontentsline{toc}{chapter}{Introduction}
\subsection*{Motivation}

Although the Moore's law\footnote{https://en.wikipedia.org/wiki/Moore's\_law} still applies and the performance of computers is now greater than it ever was, there still exist some tasks which require a lot of computer power. Despite parallelism, their accomplishment takes a non-trivial amount of time. Among these tasks are for example computations of some special prime numbers, processing large data sets and many others.

It's natural to ask, how we can speed up such computations. Obvious possibility is to get better hardware, however this is limited by technical progress and can be quite expensive. Widely used approach is to parallelize the computation, that is different parts of the job are run concurrently (in parallel) which results in reduction of the processing time. Generally, tasks can be parallelized on different levels. We will talk about parallelism on the whole computers level, i.e. each computing node is represented by a standalone machine. These nodes are independent and their physical distance can be arbitrary big. The only condition which has to be fulfilled is, that the nodes are connected together and form a network. It means that they can communicate and exchange data. This approach makes sense only with sufficiently big tasks, because the network communication is relatively slow and for many tasks it would be a bottleneck. There are also different ways how the general term "parallelism" can be understood. We will consider so-called data parallelism. That is, each computing node runs the same program, only the data differs. This is the main difference from the task parallelism concept in which different nodes can run completely different tasks.

Obviously, not all tasks can be parallelized easily or even are not possible to parallelize at all. On the other hand, there exist also jobs that can be split into smaller parts, each of which can be processed independently and the results then can be joined together in the end. Concurrent processing of such jobs is quite straightforward and allows to use of the data parallelism. These jobs are suitable for our framework.

\subsection*{Goal of the thesis} % (fold)
The main goal is to implement a robust framework which should be able to split the given job and  distribute the work among the participating nodes. Each node then processes the assigned parts and the results are eventually collected and joined together. The domain of suitable jobs is restricted to those that are described in the preceding paragraph.

The concept of parallelization has been known for many years and many different approaches have been introduced. Some of them are discussed in \hyperref[existing-solutions]{chapter 1}. Nevertheless, the great majority of them use some kind of hierarchy or need some control node, whose functionality differs from the others. Our main motivation was to provide solution, in which all the nodes are equivalent. Furthermore, each can initiate the process or serve as a computing node. There also exists one node which controls the process while the rest is working. But the point is, that every single node can become the initiator (master) or serve as a slave or even both at the same time. Our framework was implemented with the application for the video encoding in mind. However its essence is generic, so it could be easily modified to be used with an arbitrary job which fulfills the conditions mentioned above.

The purpose is to implement framework which conforms to the peer to peer paradigm as much as possible. We also want to keep the logic in the node which have demands to be processed in order to ensure better control and more deterministic behavior. Another important goal is to minimize occurrences of errors in the system and provide good ability to recover from them. The system is also intended to use personal computers as computing nodes, so the program typically runs alongside other applications.


% section goal_of_the_thesis (end)
\subsection*{Application to the video encoding}
Video files are basically sequences of images, so-called frames. When the file is played, the images are showed sequentially one by one. They are changed many times per second. so the illusion of smooth video is made. There are many different ways how to code the video into the digital form. Because of this, the need arises to convert the files from one format to another. The program which codes the analog video data to digital or decodes it the other way around is called codec. Some of the well known codecs are for example \textit{H.264}\footnote{https://en.wikipedia.org/wiki/H.264/MPEG-4\_AVC} or \textit{MPEG-2}\footnote{https://en.wikipedia.org/wiki/MPEG-2}. In order to convert the file, each frame has to be re-encoded, which takes quite a large amount of time. This makes the video encoding ideal task for our framework.

Before we use it, some more details have to be revealed. Because we usually want not only to watch the video but also listen to some sound, the video files are usually accompanied by one or more audio files. All of them are packed together in some container. When we want to convert the video, the container has to be opened, the video extracted, processed and stored again. The audio files are not our concern and we will simply copy them. Common container formats are \textit{avi}\footnote{https://en.wikipedia.org/wiki/Audio\_Video\_Interleave} or \textit{mkv}\footnote{https://en.wikipedia.org/wiki/Matroska}.

There is another important thing. In order to reduce file size and decrease the requirements of rendering the video, the video file actually contains only few frames which carry the whole information. These are usually referred to as the key frames. The rest of them contains only information how the frame differs from the previous one. This approach greatly reduces the file size and preserves good ability to recover from errors, e.g. when some part of the file is damaged. Consequence of this fact is, that we cannot split the video file at an arbitrary place. This is because if the chunk started with some non-key frame, the conversion would not be done properly.

Issues connected with video encoding are quite complex and used algorithms are very sophisticated and advanced. W edid not reinvent the wheel and used third party software for the video encoding, namely the ffmpeg and ffprobe from the FFmpeg\footnote{https://www.ffmpeg.org/} project. For the sake of simplicity, the program outputs always video files packed in the Matroska container. The choice of the codec is arbitrary, however it is restricted by the list of options. All of this software is free and published under the GNU General Public License\footnote{https://en.wikipedia.org/wiki/GNU\_General\_Public\_License} or GNU Lesser General Public License\footnote{https://en.wikipedia.org/wiki/GNU\_Lesser\_General\_Public\_License} in case of FFmpeg.

\subsection*{Terminology}
To prevent misunderstandings there is a short list of frequently used terms:
\begin{itemize}
\item \textit{master}(or \textit{initiating node, initiator}) is a node which has job to be done, i.e. has file to be encoded. It initiates and controls the process.
\item \textit{task}(\textit{job}) refers to work that should be done, specified by the initiator.
\item \textit{chunk} denotes one part of the split file, which is supposed to be encoded independently.
\item \textit{(computing) node} is one particular entity which communicates with the master and encodes chunks for it. Theoretically, there can be more such entities on one physical computer, because each node is determined by its communicating port and address.
\item \textit{neighbor} is called every node which is in the list of the node we are considering.
\end{itemize}

\subsection*{Thesis organization}
In the first chapter is given overview of the existing technologies and approaches and the description of our framework. Second chapter describes implementation details. Third chapter introduces some experiments and summarizes the results. The fourth chapter is dedicated to alternatives and possible future improvements. In the appendix is then described installation and use of the program.
\chapter{Analysis}\label{analysis}

\section{Known approaches}\label{known-approaches}

The idea of distributed computing was here since the 70's. Different
approaches have been tried since then. Some of them are described in
this section in order to show different concepts.

\subsection{Client/server architecture}\label{clientserver-architecture}

It's very common example of centralized system. In this pattern is one
central node, known as the server and the rest of nodes, clients, are
all connected to it. All the information and important computations
happens at the server. The communication is typically initialized by the
client, which sends a request, the server processes it and returns the
desired result. Pros of this approach are relatively simple
implementation and easy maintaining of the traffic. Also the data flow
can be controlled easily. However the huge disadvantage is, that the
server is by definition a single point of failure, i.e.~if the server
goes down, the whole system becomes unusable. Also, thanks to the
asymetric nature of the system, there usually has to be more parts of
the software, one to run on the server and the other to run on the
clients. The system is dependent on the server's performance too.

\subsection{Peer to peer
architectures}\label{peer-to-peer-architectures}

\subsection*{Pure peer to peer network}

Peer to peer paradigm is virtually the opposite of the previous model.
Each node is completely equivalent to each other, can join the network
at any time as well as leave it. This approach means great scalability
and fault tolerance. On the other hand it requires more complex design
and can encounter problems with security and management of the network.
Our framework implements this model.

\subsection*{Hybrid peer to peer network}

This approach combines the previous two. There exists one node which is
dedicated to serve as a control server. Each node contacts this server
and all the control information are goes through it while the data are
exchanged in the same manner as in the pure peer to peer networks. This
helps to get rid of problems with management and provide information
coherence while scalability is preserved. However there is reintroduced
the issue with the single point of failure.

\subsection*{Super peer architecture}

Another try to improve the architecture considers the concept of the so
called super peer. It is a computer, usually with slightly better
performance than the other nodes, that plays the role of the control
server, but only for some group of nodes. This forms some kind of
autonomous groups (clusters) and has the same advantages as the hybrid
network but the potential super peer failure is not so crucial. Also
each super peer can have backup so the robustness of the network can be
very good.

\subsection{Summary}\label{summary}

We have shown different implementations of the idea of distributing the
work. Each have its own specifics and may be suitable for some
application. We have chosen to implement our framework to behave like a
pure peer to peer system as much as possible. Although the super peer
architecture offers better control, we decided not to use it for several
reasons. Mainly because our framework is supposed to be used in rather
small networks where this paradigm could be quite exaggerated. Also we
did not use the hybrid network because we definitely wants to avoid
presence of the single point of failure.

\section{Existing solutions}\label{existing-solutions}

Since encoding of the video files is quite reasonable task, there was a
few implementations of the similar issue. For example D. Hughes and J.
Walkerdine from the Lancaster university published in their paper a
solution which is using Lancaster's P2P Application Framework. They
implemented a Java plug-in for this framework which uses Microsoft
Windows Media Encoder SDK. Their approach was quite similar to ours and
they achieved quite persuasive results. However their solution was
usable in the specific environment only.

\section{Framework description}\label{framework-description}

\subsection{Basic overview}\label{basic-overview}

The heart of the program is one executable file, that should be
accompanied by the configuration file. This is discussed in detail in
\hyperref[installation-and-use]{chapter 2}. Main options that should be
set are IP address and a number of port on which the program should
listen. It is also essential to provide credentials (i.e.~address and
port) of some node which should be contacted by default. When more nodes
are spawned, the network is formed and the computation may begin. Note
that information about neighbors spread in a nondeterministic manner, it
matters who is contacted by whom. So it's good to have one or more nodes
which serve as super peers, i.e.~the rest contacts only these nodes.
Once the network is established, the computation can begin. If one (or
more) nodes have tasks to be done, it can start the process. The file is
then processed and divided into chunks in the initiating node. These
chunks are distributed among neighbors, processed and returned back.
Once the initiator has all chunks back, it joins them together and the
process ends.

\subsection{Neighbor maintaining}\label{neighbor-maintaining}

Each node maintains the list of its neighbors. The list is refreshed
periodically, so the node keeps track of the current network state.
Besides this main list exists another list, which contains potential
neighbors. Generally each node that has communicated with the given one
sometime in the past is added to the list of potential neighbors. The
purpose of this list is to reduce amount of time spent with maintaining
the neighbors list. That is, no more than required count of neighbors is
maintained.

Neighbors are uniquely identified by the pair of address and
communicating port. This is sufficient for the potential neighbors,
because before the neighbor is added to the main list, it has to be
contacted. Additional info is then added so more complex structure is
needed for storing neighbors. This structure contains information about
neighbor's quality, last known state etc. The quality of the neighbor
helps to prefer one neighbor to the other when picking the one to
contact. It is updated after each chunk delivered from the given
neighbor and reflects the neighbors computation power together with the
speed of the connection.

\subsection*{Initialization and discovery}

The node is provided with the address and port of the neighbor which
should be contacted by default, that is, when the neighbors list is
empty. Each node has minimum count of neighbors which should be in its
list. This number is checked periodically. In case that the list is too
small, the list of potential neighbors is checked. If its empty, the
node has to obtain more neighbors so it picks one neighbor from the
list, contacts it and receives some suggestions. If the list is empty,
the default node is contacted. Once the suggestions are received, the
node adds the new addresses to the list of potential neighbors. When the
list contains some potential neighbors, the node can contact them and
add as regular neighbors.

\begin{figure}[h]
\begin{center}
\includegraphics[scale=0.40]{./img/workflow_neighbors.pdf}
\caption{Maintaining the neighbors}
\end{center}
\end{figure}

\subsection*{Gathering neighbors}

When the node has not enough neighbors, it can use another mechanism.
This mechanism uses a flood technique to spread the request among the
nodes in the network. The request is send to each neighbor, which
spreads it further in the same manner. However the request is equipped
with time to live value which is decreased after each hop so it does not
spread forever neither too far. Once some node receives the request and
is not busy, it contacts the initiator directly, so it can add him to
the list of potential neighbors and then possibly as the regular
neighbor.

\subsection*{Withdrawing from the network}

When the node wants to leave the network it has to abort the process, if
it is initiator. Then a special message is sent to each neighbor, which
informs them so they can react accordingly. That is, if the neighbor has
tasks to process for the leaving one, it removes them from the queue and
then removes the neighbor itself. Removed neighbors are completely lost,
they are no longer stored in either of the lists. The process of saying
goodbye to neighbors is asynchronous, the neighbors do respond with some
acknowledgement message, but the leaving node does not wait for it,
because it could cause potential deadlock in case that the other node is
unable to respond for some reason.

\subsection*{Neighbor's failure}

There is no guarantee that all the neighbors leave the network properly.
The program itself can encounter error or be terminated violently.
Another possibility is some unpredictable error of physical character,
for example power failure, network problem etc. In those cases it's
essential for the other nodes in the network to be informed about this
fact. Especially it's very important for the initiator who had some
tasks processed by this node. In order to ensure handling of this
possibility the neighbor list is checked periodically. If some neighbor
does not respond, it is removed fro the list and all the data connected
with it are treated accordingly. Namely the chunks assigned to it are
resent.

\subsection{Distribution of chunks}\label{distribution-of-chunks}

Once the file is splitted, the chunks has to be distributed, processed
and finally collected. To achieve this, we must deal with several
issues, which are described further.

\subsection*{Life cycle of the chunk}

Firstly, we have to keep track of every chunk's state. That is, we have
to know whether the chunk is waiting in the queue, has been sent to
process or have returned already. For easy manipulation each chunk is
represented by the dedicated structure, which holds information about it
as well as information essential for transfer. This structure is further
described in \hyperref[implementation]{chapter 3}. From now on we will
use the term chunk for both the physical file and the reference. Typical
chunk's life cycle looks like this: The chunk is created and pushed to
the waiting queue. Later it's popped out and transfered to the
processing node. There it is enqueued for processing, then processed and
sent back. Meanwhile the initiator holds the reference in the list of
tasks being processed. In case of failure of the processing node, the
chunk is pushed to the waiting queue again. Also when the chunk waits
for return, it's checked periodically and if the computation takes too
long, it's resent too. Note that this can cause the situation, when one
task is being processed by more than one neighbor. However it's not a
problem at all, because if the chunk returns more than once, it simply
is not accepted, because it is not needed anymore. Once the chunk
returns successfully, the reference is moved to another list, where it
waits for completion of the task. When all the chunks are collected, the
joining process may begin and the task execution ends.

\begin{figure}[h]
\begin{center}
\includegraphics[scale=0.40]{./img/workflow_chunks.pdf}
\caption{Processing a chunk - initiator part}
\end{center}
\end{figure}

\subsection*{Storing files}

Tightly coupled with this process is the problem of storing the file.
During the processing of each chunk four files has to be created.
Firstly is created when the original file is splitted. This file can't
be removed until the processed chunk returns, because it has to be
available in case that the conversion fails for some reason. Another two
files are created at the processing node, one for the input and one to
store the output. The last file is created at the initiating node again
to hold the processed chunk. This means, that the initiator has to has
free disk space at least two times the size of the resulting file.

\subsection*{Picking neighbors}

Last but not least we have to choose policy to whom the chunks are
distributed. We want to achieve as big speedup as possible, while
preserve rather small list of neighbors. When the chunk is popped out
from the queue, the initiator looks for suitable neighbor. That is the
neighbor has free status in the corresponding structure. If no such
neighbor is found, the chunk is re-queued and another try is postponed.
Also the gathering process described in the previous section begins. If
some neighbor is available, the chunk is assigned to it and the transfer
may begin. This mechanism may seem odd, but if it was the other way
around, it could possibly cause problems. If free neighbor was chosen
first, the process could block afterwards, because of the lack of
chunks. Till some chunk become available, the picked neighbor's state
could change. So it is necessary to process the actions in this order.
The initiator keeps track how many chunks were send to the particular
neighbor and it does not send more than specified count to one neighbor
because it could potentially lead to delay. The flag indicating whether
the neighbor is free helps to control the flow. Each time chunk is
assigned to the neighbor, the flag is set to false value to prevent
sending more chunks in parallel. It's set to true again after the
successful completion of the transfer. When the neighbor is too busy, it
can express it in the communication, so the flag is set to false to
prevent overloading of the neighbor. The flag is also refreshed during
every periodic check.

\subsection{Security issues}\label{security-issues}

The present implementation is possibly vulnerable to some security
threat. This is caused partly by the pure peer to peer nature, because
it is difficult to control traffic and authorize all nodes in dynamic
environments like this. It also was not our aim to solve this issue. The
framework is supposed to be used mostly in LAN's where all the peers are
trustworthy. Otherwise it could be compromised easily. For example when
the encoded chunk arrives, it is not checked whether it has been sent to
this node or not. So a malicious chunk could be infiltrated causing bad
output or even failure of the joining process.

\subsection{Networking handling}\label{networking-handling}

The network communication is the most important part of the framework.
Standard C library functions and structures were used which are
conforming to POSIX.1-2001 standard. Although the program is intended to
be used on the UNIX or UNIX-like operating systems, it should be
portable to the Microsoft Windows systems as well thanks to the use of
this standard. To preserve simplicity, all the network communication
makes use of the TCP protocol. The system primarily uses the IPv6
addresses, but it can be run in the mode which uses addresses of the
IPv4 family only. Further and more detailed information can be found in
\hyperref[implementation]{chapter 3}.

\subsection{User interface}\label{user-interface}

To provide interaction with the user, the curses API is used. This
library makes it possible to control the terminal screen. That means,
the application does not require any special GUI libraries and is able
to run interactively even on machines without the X server. The control
is rather simple, offering possibilities to load the file, start or
abort the process, show information about neighbors and so on. The
interaction require only a keyboard, no mouse is needed at all. Concrete
information together with some examples can be found in
\hyperref[installation-and-use]{chapter 2}. Detailed information about
the implementation, namely the synchronization problems are discussed in
\hyperref[implementation]{chapter 3}.

\subsection{Problems, alternatives and possible
improvements}\label{problems-alternatives-and-possible-improvements}

Some alternative approaches were also considered during the designing.
One of them was to don't use periodical checking at all. It was based on
the idea, that there would exist permanent connection with each neighbor
and the change of state would be indicated by the events related to this
connection. However it was rejected due to the requirements connected
with keeping the connection. Furthermore, the changes of ready state of
the node would have to be check either periodically or the node would
have to inform all its neighbors (even potential) about each change
which would lead to another problems to deal with and is in contrast
with passivity of the slaves anyway.

Another issue was whether use some more sophisticated way of
distributing the chunks. Namely some kind of hierarchy was considered
when the chunk references would be distributed to neighbors in packs,
where it would be further splitted and distributed among neighbor's
neighbors and so on, so a kind of a tree structure would be formed. The
transfer of the file would be processed directly between the initiator
and the leaf node. However this approach turned out to be complicated
and brings many problems. For example in case of failure of some node
which is high in the hierarchy a lot of chunks would have to be
re-distributed. Also the initiator's ability to control the distribution
would be reduced. Moreover, the advantages of this approach are not so
significant at all, because the biggest portion of time is spent during
transfers and processing the chunks and the time spent with distribution
of references is not important at all.

Also the current implementation creates a separate connection for each
data transfer. Alternatively, each chunk could be delivered and returned
using the same connection which would lessen the demands of the
communication. The connection's termination would also indicate problem
with the chunk's processing or the neighbor itself. But the connection
termination does not necessarily mean the failure of the process.
Because it is desirable to avoid needless re-encoding of the chunks,
this situation would has to be treated specially which would introduce
additional problems. Also this approach does not fit very well to the
current design which uses system of commands.
\chapter{Installation and use}\label{installation-and-use}

\section{Download}\label{download}

First it's essential to get the source code. You can clone the code
directly from the git repository using command

\begin{verbatim}
#git clone https://github.com/vojtsek/VideoCompression.git
\end{verbatim}

Alternatively you can download the zip file and unpack it in some
directory.

\section{Requirements, installation and first
run}\label{requirements-installation-and-first-run}

To run the program successfully it's essential to have \textit{ffmpeg}
and \textit{ffprobe} installed on your computer. Although technically it
doesn't matter which codec is used, the program currently uses only
H.264 standard for encoding of the output, so it assumes the
\textit{ffmpeg} has been compiled with the \textit{x264} codec.
Otherwise the program does not have any special requirements except
standard libraries which should be available on all UNIX systems, so
once you installed these programs, you can change to the directory
containing the source code and run the installation script:

\begin{verbatim}
#cd DVC
#./install.sh
\end{verbatim}

The installation script is a regular Bash script, so the Bourne again
shell interpreter is required to run it successfully. It explores your
computer, i.e.~gets the IP address, finds location of \textit{ffmpeg}
binaries etc. Then it creates home directory for the program. The home
directory contains data of the program's run. These include intermediate
results as well as the final result and log files. The installation
continues with generating the configuration file. This file is crucial
for the framework. Before setting each option, the script prompts you
for confirmation of the value. If you type nothing and just press the
Enter key, the suggested value is used, otherwise the script uses your
input. The resulting file is stored in the \textit{bin} directory which
is created during the installation. It's a plain text file so you can
edit it anytime in the future. The script then continues with
compilation of the program. If everything is all right, you can change
to the newly created \textit{bin} directory and continue. The directory
\textit{bin/lists} contains some supporting files that should not be
change, otherwise the program could behave improperly.

The last step before you can run the program is to check the
configuration. The configuration is saved in the \textit{bin/CONF} file,
which is created by the installation script. It's important to provide
valid path to the \textit{ffmpeg} and \textit{ffprobe} executable and
address with port of the neighbor that should be contacted initially.
Otherwise you won't be able to join the network. The field \emph{MY\_IP}
is not essential as long as the initial neighbor is alive - it will be
recognized automatically.

Then you can finally run the program. Some of the settings can be
changed by providing options, the available ones are listed below.
\pagebreak

\begin{itemize}
\itemsep1pt\parskip0pt\parsep0pt
\item
  -s \ldots{} no address will be contacted initially
\item
  -n address\textasciitilde{}port\_number \ldots{} node to contact
  initially
\item
  -a address\textasciitilde{}port\_number \ldots{} address and port to
  bound to
\item
  -h directory \ldots{} path to the home directory
\item
  -i file \ldots{} file to encode
\item
  -p port \ldots{} listening port
\item
  -d level \ldots{} debug level
\item
  -q quality \ldots{} quality of encoding
\end{itemize}

If the string \emph{ipv4} appears among parameters, the program will use
only the IPv4 addresses, in which case should be the \emph{CONF} file
changed appropriately.

\section{Using the program}\label{using-the-program}

When you run the program, the initial screen appears. You can choose
desired action then using function keys. You can see the initial screen
in the figure below. Available options are highlighted.

\begin{figure}[h]
\begin{center}
\includegraphics[scale=0.35]{./img/init-screen.pdf}
\label{initial-screen}
\caption[initial-screen]{Initial screen after joining.}
\end{center}
\end{figure}

The important key bindings are listed in the table below.

\begin{table}[h]
\begin{center}
 \begin{tabular}{ | l | c |}
   \hline
   F6 & Show information \\ \hline
   F7 & Start the process \\ \hline
   F8 & Load the file \\ \hline
   F9 & Set values \\ \hline
   F10 & Abort the process \\ \hline
   F12 & Quit the program \\ \hline
   Up, Down & Traverse available options \\ \hline
   Enter & Confirm the input \\
    \hline
 \end{tabular}
 \caption{Table of the control keys.}
 \end{center}
\end{table}

First you should load the video file. When the corresponding function
key is pressed, the program prompts you for the file location. You can
type in the absolute path of the file. Once you use the file, it is
stored in history which you can browse using up and down arrow keys.

\begin{figure}[h]
\begin{center}
\includegraphics[scale=0.35]{./img/loading.pdf}
\label{loading-files}
\caption{Loading the file.}
\end{center}
\end{figure}

Then you can set some parameters or show different information using F6
respectively F9 function keys. These keys provides set of options which
you can choose from. When you are satisfied with the settings, you can
start the process. The program then starts splitting the file and
distributing the chunks while informing you about the progress. When the
process is done, the file is joined and you can do further actions.

\begin{figure}[h]
\begin{center}
\includegraphics[scale=0.35]{./img/process-initiator.pdf}
\caption{Overview of the process.}
\end{center}
\end{figure}

\begin{figure}[h]
\begin{center}
\includegraphics[scale=0.35]{./img/processing.pdf}
\caption{Processing of the tasks.}
\end{center}
\end{figure}

\begin{figure}[h]
\begin{center}
\includegraphics[scale=0.35]{./img/joining.pdf}
\caption{Joining the file.}
\end{center}
\end{figure}

To obtain more information about what is going on, the \textit{-d}
option may be used which allows to specify level of debug messages that
will be shown.
\chapter{Implementation}\label{implementation}

This chapter describes some used mechanism in more detail. It also
introduces some classes and methods, but it is not supposed to serve as
a detailed and full documentation. The Doxygen software has been used to
generate documentation so the complete overview of the code can be found
in the attached HTML document. The program is implemented in the C++
language. This allows the use of standard C library functions. Also the
functionality provided by the C++ STL is exploited. Some of its
containers are used as a base for containers with synchronized access
implemented in the framework. Some of the functionality from the C++11
standard is also used, so the use of the program is limited to computers
with compiler that supports this standard.

\section{Networking}\label{networking}

One of the most important issue is how to handle the networking. The
chosen approach will be described in this section. As it has been said
already, the program uses TCP for any kind of network communication.
This is certainly good option when we want to handle data transfers,
however, it can be considered unnecessarily demanding for simple tasks.
However to preserve the implementation simple, we chose not to use UDP.
At least the advantages of the use of TCP are exploited. To utilize the
possibilities of the operating system, standard socket API is used.
Information about addresses are stored in the \textit{sockaddr\_storage}
structures which are suitable for storing both IPv4 and IPv6 addresses.
This approach makes it possible to switch between both the versions
easily. There are also some helper functions to work with address
structures which are generic in the use of an address family. To provide
easy manipulation with addresses, the structure \textit{MyAddr} was
created which groups the related functionality together.

\subsection{Spawning connections}\label{spawning-connections}

The ultimate class to handle the connections is the
\textit{NetworkHandler} class. It provides all the necessary
functionality. Each instance of the program runs separate thread which
binds to the given listening port and starts accepting connections. When
the connection is accepted, new thread is spawned to handle the
connection. When the program wants to make the connection, it provides
structure referring to a given neighbor together with the commands to
execute. to the \textit{NetworkHandler} instance. The connection is then
spawned and handled. Here we encounter the topic of commands. Each
action consists of set of commands that implements it. Commands are
instances of a given class. Each command has two parts. When one node
initiates the connection, it provides the vector of commands to be
executed. They are processed in the loop in the following manner: The
command's \textit{execute} method is invoked. It communicates over the
network. Firstly it sends name of the command to be invoked in the peer
node. The peer's thread loops too. First it reads the name of the
command and then it invokes it. In that moment there are command methods
running in both nodes and they can communicate. When they end, the
receiving node waits for another action while the initiating node spawns
next command from the vector, if any is present. Otherwise it ends the
connection. This mechanism is used to handle all the network
communication. What happens in case of problems is described in the
section about handling errors.

Incoming connections are always handled asynchronously. Nevertheless,
the outgoing connections may be handled in the synchronous way. In can
be essential sometimes. For example if the node needs to obtain some
potential neighbors because his list is empty, it has to wait until the
action ends, because if it just sent the request and continued, it would
probably find the list still empty.

\subsection{Protocol}\label{protocol}

As it was said, the communication happens thanks to commands. Thread
that handles the incoming connection is provided with respective file
descriptor and address of the communicating node. First data that
appears are considered as the listening port number of the node, so it
can be identified in the neighbors list or added to the list of
potential neighbors. Then the name of the command is sent, which is
represented by the \textit{enum} type. If no error occurs, the
confirmation is send and the appropriate command's method is invoked.
After returning from the method, another command is read. If there are
no data left, the connection is closed.

\subsection{Transferring the data}\label{transferring-the-data}

First problem every network application has to deal with considers the
byte order. The POSIX sockets API provides set of functions to deal with
it. Namely it's \textit{htons} and \textit{ntohs}, or \textit{htonl} and
\textit{ntohl} respectively. These functions convert the native
representation of short (long) data types to the network byte order. The
framework uses these function.

\subsection*{Base transfer functions}

Each of the following functions has basically two parts, sending and its
receiving counterpart. Integers are stored as the \textit{int64\_t} type
which ensures correct communication even between 32 and 64 bit nodes.
Then they are sent using the mentioned converting functions. When the
string is transfered, first is sent the length of the string, followed
by the appropriate number of characters. Commands are sent as numbers,
wrapper functions are used which converts between the \textit{enum} type
and \textit{int32\_t} explicitly. Sending the structures containing the
addresses is managed by another special function. It converts the
address to strings and sends it in this form, followed by the port
number. This allows to handle both IPv4 and IPv6 addresses, the format
is recognized during the reversed conversion.

\subsection*{Transfering files}

The most delicate task is to transfer the files. The file is first
check, and its size is determined. Then it is sent and then the function
repeatedly reads part of the file to buffer and sends it until all file
is processed. The count of sent bytes is compared with the actual file
size in the end. The receiving side accepts the bytes and writes it to
the file continuously, the file size is checked in the end. The data are
first written to a temporary file and after the successful transfer it
is renamed. This mechanism prevents inconsistency of the received files.
Each file is referenced by the
\hyperref[the-transferinfo-structure]{structure}. Among other things
this structure stores counter of unsuccessful sent tries. If the counter
exceeds given limit, the neighbor is treated as invalid and his state is
set to non-free. The file is also checked, if it is not valid for some
reason the whole process has to be aborted because there is no way how
to fix one specific file.

\subsection{Handling errors}\label{handling-errors}

Unfortunately the network environment is quite error prone and all the
action has uncertain results. Moreover, the communication can be
interrupted at any time. Because of this it is important for the network
application to be able to deal with different error situations.

\subsection*{Errors during the connection handling}

Almost all the functions indicates error by the negative return code.
These codes are checked so the error can propagate. If some error
happens during the control communication, the loop handling the
connection simply brakes, so the connection is closed. The commands are
invoked in the try-catch block, so if the data have been corrupted or
the synchronization has been lost, the next invalid command name raises
an exception and the error is handled. The loss of the synchronization
may be detected thanks to the obligatory confirmation of every command.

\subsection*{Other errors}

If an error happens during the file transfer, the receiving side detects
inconsistency thanks to checking of the file size, so the bad file can
be removed. Generally, if any error is encountered during the
communication, the corresponding execute method indicates it by its
return value, so it can be propagated further. The signal handler also
has to be set to cover situations when the connection is destroyed
unexpectedly and the SIGPIPE is delivered.

\section{Structures' overview}\label{structures-overview}

This section describes two main structures used in the framework. They
common sign is, that they inherit from the Listener class, so they have
to implement the invoke method. This fact makes it possible to use them
as \hyperref[periodic-actions]{periodic listeners}.

\subsection{The TransferInfo
structure}\label{the-transferinfo-structure}

This structure serves for referencing the chunk. It contains flags used
for transfer, addresses (source and destination), information about the
video and path locating the physical position of the file. This field is
important because it makes it possible to reference the file. It is
changed during the transfers. The structure also contains information
which help to determine the encoding process as well as statistics
describing the result. These are used to compute and update the quality
of the neighbor. The structure is also equipped with a pair of methods
that transfers it over the network, which simplifies its usage. Periodic
invocation decreases the timer. It is used when the referenced chunk is
waiting for return. For the first time the timer reaches zero the
neighbor is checked, if it is alive. If yes, the timer is send one more
time. If the neighbor doesn't respond or the timer reaches zero for the
second time, the chunks is resent.

\subsection{The NeighborInfo
structure}\label{the-neighborinfo-structure}

Instances of this structure are kept in the
\hyperref[neighborstorage]{NeighborStorage class}. They represent the
neighbor, that is its address and listening port. It also keeps the
information about quality of the neighbor. The time from last check is
stored too. Periodic invocation causes the timer to decrease and
possibly contact the neighbor to refresh the state.

\section{Important classes}\label{important-classes}

\subsection{The NeighborStorage class}\label{the-neighborstorage-class}

It is class used for storing the \textit{NeighborInfo} structures. It
provides several methods to maintain neighbors list while preserving
synchronization. This is crucial, because the information about neighbor
can change any time but it's desirable to keep our knowledge consistent.
The class has one instance and helps to keep the information about
neighbor at one place, so the manipulation can be controlled.

\subsection{The NetworkHandler class}\label{the-networkhandler-class}

This class is used to handle all the networking issues. It provides
functionality to spawn the connections or contact neighbors. It also
hold the list of potential neighbors. This list differs from the
neighbors list, because besides address and port, which are necessary,
no further info is stored about potential neighbors. Also, most of the
other classes have a notion about this list. When the lack of neighbors
appears, simply the function \textit{obtainNeighbors} is invoked, which
uses the list internally. This class also handles adding of the new
neighbors.

\subsection{The Data class}\label{the-data-class}

It is a singleton class which helps to keep all the data at one place
while accessible from anywhere in the program. It holds the instances of
the \textit{NeighborStorage}, \textit{State} and all the
\hyperref[queues]{queues} used during the transfer and the encoding
process.

There are more significant classes such as TaskHandler and
WindowPrinter, which will be discussed in the corresponding sections.

\section{Periodic actions}\label{periodic-actions}

The framework uses a mechanism which invokes some actions periodically.
Pros and cons of this approach are described in the respective sections,
namely
\hyperref[problems-alternatives-and-possible-improvements]{the alternatives}.
To implement this mechanism, separate thread runs that loops and once
after each time quantum it invokes methods of structures that inherits
from the \textit{Listener} abstract class and are stored in the special
queue. The time quantum is defined as a constant, so all timers actually
express count of the quanta left. Obvious disadvantage of this approach
is busy waiting that is used in the loop. Alternative approach could use
a signal handler and setting an alarm. However, this could lead to not
necessary asynchronous interrupts. Since the mutexes are used to avoid
race conditions, a problem could occur if the signal interrupted some
method holding a ``bad'' mutex.

\section{Queues}\label{queues}

As it is described in the corresponding
\hyperref[distribution-of-chunks]{section}, the references to chunks are
stored in different queues, depending on the state in which they are.
Since the whole process is non-deterministic, different conditions may
occur and more than one thread could need to work with the queue at one
moment. This means, that some way of synchronization has to be provided.
Because of this textit\{SynchronizedQueue\} has been created. It is a
kind of a lock-free structure. Basically it provides usual functionality
that could be requested from the queue, but ensures avoiding race
conditions because only one thread at a time can access the underlying
data. This is implemented by the use of textit\{mutex\}. The
\textit{pop} method also uses the conditional variable, so if there are
no data that can be popped at the moment of invocation it blocks and
waits for a signal.

\section{Working with input and
output}\label{working-with-input-and-output}

All the file operations are customized for use on the UNIX operating
system. Nevertheless, there is possibility to implement respective
functions to work on different operating systems easily.

\subsection*{File operations}

Each chunk is stored in a separate file. For this reason several
functions were implemented to allow easier work with the file system.
These include functions for manipulation with the filenames, controlling
files and working with directories. Also there is a generic function
which spawns an external process. This function uses process forking,
spawns the desired process and returns contents of its standard output
and error streams. The function also accepts time after which it kills
the process. The result of the process' run propagates in the function's
return value so the caller can react accordingly.

\subsection*{Working with the video}

The video processing is secured by the \textit{TaskHandler class}. When
it is loaded, some useful properties are obtained thanks to the
\textit{ffprobe} program. To allow easy processing, the output is in the
JSON format which is then parsed with the help of the \textit{rapidjson}
library. This library consists of header files only and is distributed
as a part of the source code. It is available under the MIT license
which makes it suitable for usage. These values can then be showed using
the F6 key. More importantly, it is used to compute the number of chunks
that will be created. Note, that the number of chunks can change
slightly during the splitting process. It is caused by the fact, that
the teoretically computed count does not consider the positions of key
frames. So the chunks can actually have different sizes and thus their
count differs. This fact also implies, that each chunks has got slightly
different size. When the process starts, the instance of \textit{ffmpeg}
process is spawned which splits the file. The resulting files are then
stored in the special subdirectory of the working directory. For easy
identification, each process is assigned a unique code determined by the
time stamp. The chunks are then numbered in increasing order. The names
are then stored in the referencing structures that are created at this
point. Then the chunks are distributed as described in the
\hyperref[distribution-of-chunks]{section} about distributing. Each
processing neighbor encodes the chunks using the \textit{ffmpeg} and
then sends it back. The information about the chunk, such as the level
of the encoding quality are stored in the referencing structure. Once
the chunks are collected, first a list of the files to join is created
and then the \textit{ffmpeg} is used once more to join all the chunks to
the output file.

An alternative approach to splitting the file was used during the
development. Firstly the position of every split was computed. Then it
was spawned one process per each chunk. The advantage was, that the
distribution could begin after the first chunk was created and therefore
some time was saved. However this approach turned out to be bad because
of the existence of key frames. It does not allow to split the file at
the arbitrary position so certain shift was observable in the resulting
file.

\subsection*{User interaction}

To provide interaction with the user, the \textit{curses} library is
used. User can provide input from the keyboard. There is set no delay of
the input, so the buffering is disabled. This causes the need to handle
the reading manually but also makes it possible to control the input and
accept the commands immediately. The output is provided using the
\textit{WindowPrinter} class. This class stores the queue of lines of
output and provides functionality to add or remove some. Each record
holds the line together with the information of the style how to print
it. The screen is divided into four parts. Each part spans the whole
width. There is a line displaying available commands at the top. The
biggest portion of the vertical space belongs to two windows. The first
one displays different information about processing, neighbors or file
properties. The second window displays the status changes, notifications
and potentially some debugging messages. The most bottom part shows a
prompt when user input is required.

Because there is usually a lot of threads that can cause the screen to
refresh (this means the particular \textit{WindowPrinter} instance is
updated), it is important to allow only one graphical update at the
moment, otherwise it could cause inconsistency of the graphical data.
This is ensured by a mutex assigned to each \textit{WindowPrinter}
instance.

\section{Synchronization}\label{synchronization}

Because of the nondeterministic nature of the application, it is
necessary to provide some kind of synchronization to ensure data and
information consistency. This problem is solved by the use of mutex and
conditional variable available in the C++ standard library. There are
implementations of queue and map like structures which offer lock-free
access. These classes uses containers from the STL and synchronization
primitives mentioned above. These structures, namely the
\textit{SynchronizedMap} and \textit{SynchronizedQueue} are used to
store chunk references. Operations with the list of neighbors have to be
synchronized too, because for example one thread could be working with
the neighbor's reference while the other finds out that the neighbor is
not responding and wants to remove it from the list.

Another area which have to deal with some race conditions is output
which is displayed on the screen. The output is handled by the
\textit{curses} library which provides practically raw access to the
screen. This means, that if more threads try to work with the screen at
one moment, there is high possibility that they compromise each other
and nonsensical data are displayed at the output. Moreover, this
situation can also possibly result in the segmentation fault. To avoid
these situations, the data that are supposed to appear on the screen are
stored in the respective instances of the \textit{WindowPrinter} and
mutexes are used to allow only one thread to change the content of the
storage or refresh the screen. This approach has a disadvantage, that
each call of the routine that produces some output is possibly blocking

\section{Error detection and
recovery}\label{error-detection-and-recovery}

During the process, various types of errors can occur. Errors connected
with the networking are discussed in the respective
\hyperref[handling-errors]{section}. Here we will describe other errors
that could possibly happen.

\subsection*{Neighbor failure}

If an unexpected failure of some node occurs, the other nodes which has
it in their lists must react. If the communication is interrupted in the
middle, there is no way how the other node can recognize the failure, so
the command execution just fails. However this usually leads to
repetition of the command which registers the failure. Generally the
failure is noted when the try to establish a connection with the given
neighbor fails. This can occur when processing a command, checking a
neighbor or when the timer assigned with some chunk reaches zero. In
every of these situations the same function is used, so the situation is
always treated in the same way. The unresponsive neighbor is removed
from the list. If it has some chunks assigned, they are queued for send.
If it has chunks being processed by the current node, those chunks are
trashed.

\subsection*{Chunk disappearance}

After the chunk is send to be processed, it is pushed into special queue
where it is checked periodically. If the time is up, the respective
neighbor is checked. If it is responding, the timeout is set once again.
If it does not respond, or the timer reaches zero for the second time,
the chunk is queued for send. Also, in case of the neighbor failure, the
neighbor is removed. This can lead to a situation, when one chunk is
being processed by two different neighbors at the same time. However,
after it returns for the first time, it is put into the dedicated
storage, so if it returns afterwards, it is simply rejected. This
approach could be improved by introducing new action which would cancel
the chunk's encoding process at the remote node. It was not implemented
though, because it is not clear on which node the process should be
canceled. Connected with this problem is the situation, when one node
receives by accident one chunk more times. The files are checked and
what is more, the transfer uses temporary files so the worst scenario
involves wasteful encoding of the chunk for the second time. Another
issue which has to be solved is how to set the timeout. When the
neighbor is involved for the first time, we have no information about
its performance, so the timeout is set respectively to the size of the
chunk, default multiplication factor is used. When the neighbor has
already quality factor assigned, it is used to compute the timeout. So
the quality coefficient can be seen as time needed to encode and
transfer some unit of data.

\subsection*{Other errors}

Errors can also be encountered during the manipulation with the video.
Because all the video related problems are manipulated by the external
programs, the mechanism is used, which can control the process. The
approximate upper bounds are set for each task that is supposed to be
executed and if the process' execution takes too long, the signal is
send which terminates it, thus the error code is returned.
\chapter{Experiments}\label{experiments}

The purpose of the application is to speed up the computation process,
thus it should be verified, whether the improvement makes sense or do
not. The improvement should correspond to the number of nodes involved
in the computation. What we wish is, that the dependence is of some
linear form, that is, the computation gets faster with every additional
node and it improves by the same steps. In this hypothetical ideal case
two nodes means two times faster computation and one hundred nodes means
one hundred times faster achievement of the result. However, this is
impossible for several reasons. At first, we must consider time that is
taken by the division process. More time is needed for transfers and
final join operation. Another problem raises because of the fact that
the transfers are quite demanding themselves so when more transfers are
ongoing at a particular moment, the initiator is more utilized and the
process could be slowed down due to this fact. This also implies that
the improvement does not raise constantly when adding more nodes.
Finally, we must consider that in the real situation delays can appear
due to technical reasons, network congestion or node failures.

\section{Aproach to testing}\label{aproach-to-testing}

If we want to obtain reasonable data, the measurements must be repeated
several times to prevent deviations. Also we want to keep the
measurements independent so its statistical processing is easier. Our
approach to running the tests and gathering results is described in this
chapter. Our main goal is to measure the improvement, but we would also
like to measure the impact the particular setting has on the result.

Because of the number of tests, it is desirable that the testing process
is automated. Because of that, special Bash script was used to run the
tests. The script is tailored to be used at the testing laboratory, so
it may need little modifications to work in some different environment.
It is distributed with the source code of the framework. To allow
automated and robust execution of the test, special functionality was
added to the program. It is invokable by option given at the start time
and causes the program to run in non-interactive mode, i.e.~keyboard
input is accepted, the program just processes given file and ends. This
options ?? all the essential data are given when the program is started.
To keep the measurements independent, all the instances (on every node)
of the program are started at the test beginning and they are killed in
the end. Communication with the remote nodes is handled by the ssh
program. The testing script uses a special file which describes the
particular run. Working example of such file together with explanations
of the values is given below.

\begin{samepage}
\begin{verbatim}
v6 // use IPv6
/afs/ms/u/h/hudecekv/futu.avi // location of the file to be re-encoded
2 // run the whole scenario twice
slower // quality of the encoding
10000 // chunk size [KByte]
2048576 // transfer buffer size
spawn u-pl1 2221 // spawn the program on the machine 'u-pl1', use port 2221
spawn u-pl2 2222
spawn u-pl4 2224
spawn u-pl5 2225
spawn u-pl6 2226
spawn u-pl7 2227
wait 10 // wait for ten seconds before next action
kill u-pl4 // kill the instance of program running on machine 'u-pl4'
spawn u-pl8 2228
spawn u-pl9 2229
spawn u-pl10 2230
\end{verbatim}
\end{samepage}

Thanks to this mechanism, various scenarios can be run easily without
the need of human interaction. The test data were collected by running
the test ten times for the given count of nodes. The count varied from
one to ten nodes involved. Each test was run once with chunks of 40 000
kB in size and once with 10 000 kB chunks. The same file was used each
time as well as the encoding quality. Each test stored various results,
among others the average times needed for transfer and encoding, number
of chunks, quality and count of involved nodes. Because we had not the
chance to run the tests in some dedicated network, the computation times
may vary for the given setting depending on the current conditions. The
tests showed however, that if we multiply the average time needed to
encode one chunk by the count of chunks, the product corresponds to the
time that would be taken by the normal encoding process. This allows us
to deal with the problem, because we can compare the time with this
computed estimation and the error will be minimal. The desired values
have been gathered in two ways. Some of them, for example average
transfer and encoding times, are measured directly in the program and
then outputted in special file. The script just reads it from this file.
The rest of the values is obtained in the script. \#\# Results
Measurings showed, that the dependence between the number of involved
nodes and the improvement is approximately logarithmic. To evaluate the
data, simple linear regression model was used. Specifically, subsequent
formula was used:

\begin{center}
$\frac{distributed\_time}{single_node_time} = x_1 \times log(neighbor_count - 0.9)$
\end{center}

The absolute value has been added since the model fits better this way -
in the case when one neighbor is used, the distributed computation is
actually slower. In the following figures are showed the achieved
results. The blue dashed line represents the estimate based on the
model. The data are visualized as black crosses, red squares show
respective mean values.

\begin{figure}[h]
\begin{center}
\includegraphics[scale=0.90]{./img/Rplot.pdf}
\caption{Achieved improvement - all measurements}
\end{center}
\end{figure}

\begin{figure}[h]
\begin{center}
\includegraphics[scale=0.90]{./img/Rplot10k.pdf}
\caption{Achieved improvement - 10 MB chunks}
\end{center}
\end{figure}

\begin{figure}[h]
\begin{center}
\includegraphics[scale=0.90]{./img/Rplot.pdf}
\caption{Achieved improvement - 40 MB chunks}
\end{center}
\end{figure}
 
% An example of LaTeX use (uncomment, if you wish)
% \include{example}

\chapter*{Conclusion}
\addcontentsline{toc}{chapter}{Conclusion}
Based on the experiments it can be said, that the framework works fine and it can be successfully used to speed up computation in the local network. Although the achieved speed up does not grow linearly with the increasing number of nodes, it can be quite significant. It was not tested in WAN environments, however, according to the results, the transfers of the data takes indispensable portion of the whole processing time, so the improvement depends on the network throughput. 

As is being discussed in \hyperref[Problems-alternatives-and-possible-improvements]{chapter 4}, the framework could be further improved. To achieve more effective distribution of the work, more sophisticated scheduling could be employed that would take into account network topology or each node's performance and possibly create chunks of different sizes etc. It would require some better network knowledge. Also it was showed, that some redundancy for prevention of re-computing all chunks from one particular node in the case of failure could be useful. However, this would cause worse performance and also the advantages are quite unsure. Because of the speed of the network the data transfers generally seems to be a bottleneck, so good scheduling algorithm appears to be very important, together with optimal choice of the chunk size. This leads us to an idea, that in reliable network environments it can be good idea to centralize logic to special node in order to achieve better performance and accept potential malfunctions caused by the control node failure. This control node would schedule the process for each client that would ask. Another advantage of this approach is the fact, that this node could use a knowledge of the current network state, that is, which nodes are employed and how.

Nevertheless, a lot of computer networks can suffer from unreliability and there always is a possibility of node failure. That is the reason, why our framework can be very useful, since it is able to deal with error situations and does not require any special nodes. Among video encoding, it could be easily modified to process different tasks, such as processing large data sets or images. Although the efficiency is dependent on the current conditions, mainly the network speed, the achieved speedup can be quite significant.

%%% Bibliography
\bibliographystyle{csplainnat}
\bibliography{literature}
\addcontentsline{toc}{chapter}{\bibname}

%%% Figures used in the thesis (consider if this is needed)
\listoffigures

%%% Tables used in the thesis (consider if this is needed)
\listoftables

%%% Abbreviations used in the thesis, if any, including their explanation
\chapwithtoc{List of Abbreviations}

%%% Attachments to the bachelor thesis, if any (various additions such
%%% as programme extracts, diagrams, etc.). Each attachment must be referred to
%%% at least once from one's own text of the thesis. Attachments are numbered.
\chapwithtoc{Attachments}

\openright
\end{document}
